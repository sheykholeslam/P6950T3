\documentclass{article}
\usepackage[utf8]{inputenc}
\usepackage[english]{babel}
\usepackage{amsmath}
\usepackage{natbib}
\usepackage{graphicx}

\begin{document}

{\centering

\rule{\textwidth}{1.6pt}\vspace*{-\baselineskip}\vspace*{2pt} 
\rule{\textwidth}{0.4pt}\\[\baselineskip] 
{\LARGE Growing Degree-day}
\rule{\textwidth}{0.4pt}\vspace*{-\baselineskip}\vspace{3.2pt}
\rule{\textwidth}{1.6pt}\\[\baselineskip] 

\vspace{20mm} %5mm vertical space
\scshape % Small caps
CMSC 6950 - Computer Based Research Tools and Applications \\ [\baselineskip]
Extended Assignment \\[\baselineskip] 
13th June, 2016 \\[\baselineskip] 
\vspace{20mm} %5mm vertical space
Submitted by \\[\baselineskip]
{\Large Mohammad Hassan \\ Ernest \\ Mehrzad \\ Yin Zhang \\ Rufai raji \\ Lutfor Rahman \\ Mohammad\par}
\vfill
{\itshape Memorial University of Newfoundland \\ St. John's, Canada.\par} 
}

\newpage

{\centering
  \section*{Abstract}
}

{\itshape The Growing Degree Day (GDD), is a heat index that can be used to predict when a crop will reach maturity. Each day’s GDD is calculated by subtracting a reference temperature, which varies with plant species, from the daily mean temperature (we ignore values less than zero). The reference temperature for a given plant is the temperature below which its development slows or stops. For example, cool season plants, like peas, have a reference temperature of 40$^{\circ}$F while warm season plants, like sweet corn and soybeans, have a reference temperature of 50$^{\circ}$F. The total GDDs over a growing season is related to plant development. The development of plants depends on the accumulation of heat. Since cool season plants have a lower reference temperature, they accumulate GDDs faster than warm season plants. Unless plants are overly stressed by drought or pests, the total GDDs can be used to predict when a crop will reach maturity. Corn, for example, requires 1360 GDD to mature. GDDs can be computed using climatic information for any location. That computation, along with data on soil, water, and minimum and maximum temperatures, helps suggest which crops will grow best in a given region.\\ [\baselineskip] This is an extend assignment made of a set of tasks to complete. Here, we calculated GDDs based on different Canadian city's weather history. We considered three major cities, they are St. John’s, Calgary and Montreal. We collected those data from weather Canada historical data and taken the min-max temperature values year round for difference cities. Then we analyzed those data and plot the graph to illustrate the ideas and to fulfill the given tasks. 
}

\newpage
\tableofcontents
\newpage

\section{ \bf Introduction}
Growing degree-days (GDD) are frequently used as a weather-based indicator for assessing crop development. For example, many farm weather radio broadcasts or climatological bulletins convey daily or weekly accumulations as insight into the current and ongoing status of the growing season. It is important, however, that users of GDD information understand the limitations inherent with the use of this, often misrepresented, concept.\\ [\baselineskip] The status of agricultural crops is determined separately by growth and development processes. In understanding the limitations of GDD's, these 2 processes must be differentiated. Crop growth refers to an increase in crop weight, height, volume or area over a certain time scale. Development refers to the timing or progress of the crop from one stage of maturity to the next. During this progress of the crop though its phases of development, considerable variations in growth may occur. 

\section{ \bf Methodology}
\subsection{Data Collection}
\subsection{Growing Degree Day Calculation}
Growing Degree Day (GDD) are calculated by taking the average of the daily maximum and minimum temperatures compared to a base temperature, $T_{base}$, (usually 10$^{\circ}$C). As an equation: \vspace{5mm}

{\centering
$ GDD = (T_{max} - T_{min} /2) - T_{base} $\\ [\baselineskip]
}

Here, if the mean daily temperature is lower than the base temperature then $GDD = 0$.\vspace{5mm}

GDDs are typically measured from the winter low. Any temperature below $T_{base}$ is set to $T_{base}$ before calculating the average. Likewise, the maximum temperature is usually capped at 30$^{\circ}$C because most plants and insects do not grow any faster above that temperature. However, some warm temperate and tropical plants do have significant requirements for days above 30$^{\circ}$C to mature fruit or seeds.\vspace{5mm}

For example, a day with a high of 23$^{\circ}$C and a low of 12$^{\circ}$C (and a base of 10$^{\circ}$C) would contribute 7.5 GDDs.\vspace{5mm}

\[ \frac {23+12}{2}-10=7.5 \] \par

A day with a high of 13$^{\circ}$C and a low of 10$^{\circ}$C (and a base of 10$^{\circ}$C) would contribute 1.5 GDDs.\vspace{5mm}

\[ \frac {13+10}{2}-10=1.5 \] \par

\noindent
\section{ \bf Software Architecture}
\subsection{Tools Used}
\subsubsection{Programming Language}
\subsubsection{Version Control}
\subsection{Components}
\subsubsection{Makefile}
\subsubsection{Plots}
\subsubsection{Report}
\subsubsection{Presentation}

\section{ \bf Core Tasks}
\subsection{Download daily historical temperature data for several cities}
\subsection{Create a plot showing an annual cycle of min/max daily temperatures. Do this for at least three selected Canadian cities.}
\subsection{A command line program that takes arguments.}
This program should calculate the GDD. Internally your program should handle the command line arguments and implement the actual calculation as one or more functions. The output from this program needs to be persistently stored. Your choice on how to implement this storage. Later steps in your work flow must use the results of these calculations.
\subsection{Create plots showing accumulated GDD vs time for selected cities}

\subsection{Use version control (git) and collaboration tools (GitHub)}
\subsection{Create a LaTeX report summarizing the results of your project}
\subsection{Create a web based presentation for your results}
\subsection{Implement your entire workflow as a Makefile. Ensure that your entire project is reproducible}
\subsection{Create a test-suite (using the Python package nose) to demonstrate your GDD calculation works as intended}
\subsection{Project should include adequate documentation both with your source code and Readme.md file}


\section{ \bf Optional Tasks}
\subsection{Create an plot showing GDD like the example below for selected Canadian cities}
\subsection{Create a map showing effective growing degrees over both all of Canada and only for the island of Newfoundland.}
\subsection{Explore how GDD calculation depends on the choice of $T_base$. show your results for either selected cities or create maps}
\subsection{Create standalone bokeh plots embeded in your HTML presentation}
\subsection{Create a bokeh server plot so that you can look at the accumulated GDD for any city in Canada.}



\section{Conclusion}
``I always thought something was fundamentally wrong with the universe''\citep{adams1995hitchhiker}

\bibliographystyle{plain}
\bibliography{references}

\begin{figure}[h!]
\centering
\includegraphics[scale=.5]{GDD_Plot.png}
\caption{GDD Cumulative Plot}
\label{fig: GDD Plot}
\end{figure}

\end{document}
